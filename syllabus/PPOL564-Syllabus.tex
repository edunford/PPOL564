% Options for packages loaded elsewhere
\PassOptionsToPackage{unicode}{hyperref}
\PassOptionsToPackage{hyphens}{url}
%
\documentclass[
  12pt,
]{article}
\usepackage{lmodern}
\usepackage{amssymb,amsmath}
\usepackage{ifxetex,ifluatex}
\ifnum 0\ifxetex 1\fi\ifluatex 1\fi=0 % if pdftex
  \usepackage[T1]{fontenc}
  \usepackage[utf8]{inputenc}
  \usepackage{textcomp} % provide euro and other symbols
\else % if luatex or xetex
  \usepackage{unicode-math}
  \defaultfontfeatures{Scale=MatchLowercase}
  \defaultfontfeatures[\rmfamily]{Ligatures=TeX,Scale=1}
\fi
% Use upquote if available, for straight quotes in verbatim environments
\IfFileExists{upquote.sty}{\usepackage{upquote}}{}
\IfFileExists{microtype.sty}{% use microtype if available
  \usepackage[]{microtype}
  \UseMicrotypeSet[protrusion]{basicmath} % disable protrusion for tt fonts
}{}
\makeatletter
\@ifundefined{KOMAClassName}{% if non-KOMA class
  \IfFileExists{parskip.sty}{%
    \usepackage{parskip}
  }{% else
    \setlength{\parindent}{0pt}
    \setlength{\parskip}{6pt plus 2pt minus 1pt}}
}{% if KOMA class
  \KOMAoptions{parskip=half}}
\makeatother
\usepackage{xcolor}
\IfFileExists{xurl.sty}{\usepackage{xurl}}{} % add URL line breaks if available
\IfFileExists{bookmark.sty}{\usepackage{bookmark}}{\usepackage{hyperref}}
\hypersetup{
  hidelinks,
  pdfcreator={LaTeX via pandoc}}
\urlstyle{same} % disable monospaced font for URLs
\usepackage[margin=1in]{geometry}
\usepackage{longtable,booktabs}
% Correct order of tables after \paragraph or \subparagraph
\usepackage{etoolbox}
\makeatletter
\patchcmd\longtable{\par}{\if@noskipsec\mbox{}\fi\par}{}{}
\makeatother
% Allow footnotes in longtable head/foot
\IfFileExists{footnotehyper.sty}{\usepackage{footnotehyper}}{\usepackage{footnote}}
\makesavenoteenv{longtable}
\usepackage{graphicx,grffile}
\makeatletter
\def\maxwidth{\ifdim\Gin@nat@width>\linewidth\linewidth\else\Gin@nat@width\fi}
\def\maxheight{\ifdim\Gin@nat@height>\textheight\textheight\else\Gin@nat@height\fi}
\makeatother
% Scale images if necessary, so that they will not overflow the page
% margins by default, and it is still possible to overwrite the defaults
% using explicit options in \includegraphics[width, height, ...]{}
\setkeys{Gin}{width=\maxwidth,height=\maxheight,keepaspectratio}
% Set default figure placement to htbp
\makeatletter
\def\fps@figure{htbp}
\makeatother
\setlength{\emergencystretch}{3em} % prevent overfull lines
\providecommand{\tightlist}{%
  \setlength{\itemsep}{0pt}\setlength{\parskip}{0pt}}
\setcounter{secnumdepth}{-\maxdimen} % remove section numbering
\usepackage{booktabs}
\usepackage{longtable}
\usepackage{array}
\usepackage{multirow}
\usepackage{wrapfig}
\usepackage{float}
\usepackage{colortbl}
\usepackage{pdflscape}
\usepackage{tabu}
\usepackage{threeparttable}
\usepackage{threeparttablex}
\usepackage[normalem]{ulem}
\usepackage{makecell}
\usepackage{xcolor}

\author{}
\date{\vspace{-2.5em}}

\begin{document}

\begin{center}

\huge \textit{PPOL 564-01}\\
\huge\textbf{Data Science I: Foundations}\\
\Large Fall 2020

\end{center}

\hypertarget{instructor}{%
\section{Instructor}\label{instructor}}

\textbf{Professor}: Eric Dunford

\begin{itemize}
\tightlist
\item
  \textbf{Office}: Bedroom (formerly 404 Old North)
\item
  \textbf{Office Hours}: Tuesdays 9:00am to 11:00am (EST) or by
  appointment
\item
  \textbf{Email}:
  \href{mailto:eric.dunford@georgetown.edu}{\nolinkurl{eric.dunford@georgetown.edu}}
\item
  \textbf{Pronouns}: He/Him
\end{itemize}

\textbf{Teaching Assistant}: Eric LaRose

\begin{itemize}
\tightlist
\item
  \textbf{Office Hours}: by appointment
\item
  \textbf{Email}:
  \href{mailto:eml119@georgetown.edu}{\nolinkurl{eml119@georgetown.edu}}
\item
  \textbf{Pronouns}: He/Him
\end{itemize}

\textbf{Class Website}: \url{www.ericdunford.com/ppol564}

\begin{center}\rule{0.5\linewidth}{0.5pt}\end{center}

\hypertarget{course-description}{%
\section{Course Description}\label{course-description}}

This first course in the core data science sequence teaches Data Science
for Public Policy (DSPP) students how to synthesize disparate, possibly
unstructured data in order to draw meaningful insights. Topics covered
include the fundamentals of object-oriented programming in Python;
literate programming; an introduction to algorithms and data types; data
wrangling, visualization, and extraction; and an introduction to machine
learning methods. In addition, students will be exposed to Git and
Github for version control and reproducible research. The objective of
the course is to teach students how incorporate data into their
decision-making and analysis. No prior programming experience is assumed
or required.

\hypertarget{time-and-location}{%
\section{Time and location}\label{time-and-location}}

Classes will be held \textbf{\emph{virtually}} on \textbf{Wednesdays}
from \textbf{\emph{3:30pm to 6:00pm}}:

\begin{itemize}
\tightlist
\item
  August 26
\item
  September 2, 9, 16, 23, 30
\item
  October 7, 14, 21, 28
\item
  November 4, 11, 18
\item
  December 2
\end{itemize}

Holidays/Breaks/Away (No class):

\begin{itemize}
\tightlist
\item
  November 25 (Thanksgiving Recess)
\end{itemize}

\hypertarget{asynchronous-synchronous-lectures}{%
\subsection{Asynchronous \& Synchronous
Lectures}\label{asynchronous-synchronous-lectures}}

The lecture will be broken up into \emph{synchronous} and
\emph{asynchronous} components.

\begin{itemize}
\tightlist
\item
  The \textbf{\emph{asynchronous components}} will cover the main
  concepts of the lecture. These materials will take the form of
  embedded videos in class lecture notes on the course website. Students
  are required to review this content along with the lecture notes and
  readings prior to the start of class. \textbf{\emph{Asynchronous
  materials will be made available a \textbf{week prior} to the
  scheduled lecture date}.}
\item
  The \textbf{\emph{synchronous component}} will take place at the
  scheduled class time and will involve active coding walkthrough,
  breakout group sessions, and questions. The aim of the synchronous
  class time is to reinforce the concepts covered in the asynchronous
  lecture materials. Thus, it is imperative that students complete the
  asynchronous material \emph{prior to the start of the synchronous
  lecture}.
\end{itemize}

Note that this class is scheduled to meet weekly for 2.5 hours. I will
do my best to ensure that the asynchronous and synchronous material in
combination does not exceed 2.5 hours weekly. Put differently, students
will not be required to commit more than 2.5 hours to lecture. This does
not include readings, homework and/or coding discussions; rather,
bifurcating lecture materials into synchronous and asynchronous
components is necessary when learning virtually. Zoom fatigue is real,
and lectures that exceed an 1.5 hours are not effective. When we do meet
in-person, five minute breaks will be taken approximately every 40
minutes.

\textbf{All synchronous lecture material will be recorded and stored on
the class Canvas site.} Students who are unable to attend the
synchronous lecture will be able to review the materials covered in
class at a future date.

\textbf{For students attending class from afar} (i.e.~in time zones more
than 4 hours off Eastern standard time), participating in the
synchronous lecture component may not be a viable option. Please let the
professor know if you're planning on attending the course from afar.
These students will not be required to attend synchronous components of
the lecture. \emph{It is the students responsibility to review all
lecture materials and to keep pace with the course.}

\hypertarget{virtual-classroom}{%
\subsection{Virtual Classroom}\label{virtual-classroom}}

We will use \href{https://zoom.us/download}{\textbf{Zoom}} (a
web-conferencing platform) to hold class each week. Class will meet at
its regularly scheduled time each week for synchronous lectures. If you
do not have Zoom, you can download it
\href{https://zoom.us/download}{\textbf{here}} prior to the start of
class.

A link for the synchronous component of the weekly lecture along with a
link for virtual office hours is posted on the course website and
Canvas. Students will use this link to access the live Zoom call for
lecture.

\textbf{\emph{If the link breaks or does not function properly, please
check the \texttt{\#general} channel on Slack for information regarding
the new link. If there is no message regarding a new link, please
contact the professor and/or TA via Slack.}} All synchronous lecture
material will be recorded.

\hypertarget{course-objectives}{%
\section{Course Objectives}\label{course-objectives}}

The course aims to provide students with the following competencies:

\begin{itemize}
\item
  General understanding of python's object oriented programming syntax
  and data structures.
\item
  Competency using version control (Git/Github).
\item
  Learn to manipulate and explore data with Pandas and other tools.
\item
  General understanding of analyzing algorithms and data structures.
\item
  Learn to extract and process data from structured and unstructured
  sources.
\item
  Learn to use statistical learning approaches to effectively explore
  and ask questions from data.
\end{itemize}

\hypertarget{required-materials}{%
\section{Required Materials}\label{required-materials}}

\textbf{Readings}: We will rely primarily on the following text for this
course.

\begin{itemize}
\item
  \textbf{Vanderplas, J.T., 2016. ``Python data science handbook: tools
  and techniques for developers.'' \emph{O'Reilly.}} (Online version:
  \url{https://jakevdp.github.io/PythonDataScienceHandbook/})
\item
  \textbf{Miller, B. and Ranum, D., 2013. ``Problem Solving with
  Algorithms and Data Structures using Python.''} (Online version:
  \url{https://runestone.academy/runestone/books/published/pythonds/index.html})
\item
  \textbf{James, G., Witten, D., Hastie, T., \& Tibshirani, R. (2013).
  ``An Introduction to Statistical Learning: with Applications in R''.
  \emph{New York: springer}.}
\item
  \textbf{\emph{Additional readings will be posted for each class and
  can be found on the course website}}. Most reading material is open
  source and available via a link on the reading list, otherwise it can
  be found on Canvas.
\end{itemize}

\textbf{Class Website}: A class website (www.ericdunford.com/ppol564)
will be used throughout the course and should be checked on a regular
basis for lecture materials and required readings.

\textbf{Class Slack Channel}: The class also has a dedicated slack
channel (ppol564-foundations.slack.com). The channel serves as an open
forum to discuss, collaborate, pose problems/questions, and offer
solutions. Students are encouraged to pose any questions they have there
as this will provide the professor and TA the means of answering the
question so that all can see the response. If you're unfamiliar with,
please consult the following start-up tutorial
(\url{https://get.slack.help/hc/en-us/articles/218080037-Getting-started-for-new-members}).
Please follow the
\href{https://join.slack.com/t/georgetown-jt44872/shared_invite/zt-gntolqrn-BiCH6ChQTKKODMyb5PdRAw}{\textbf{\emph{invite
link}}} to be added to the Slack channel.

\textbf{Canvas}: A Canvas site (\url{http://canvas.georgetown.edu}) will
be used throughout the course and should be checked on a regular basis
for announcements, readings, and assignments. All readings and
assignments will be posted on Canvas; they will not be distributed in
class or by e-mail. Support for Canvas is available at (202) 687-4949

\textbf{NOTE: Students are encouraged to run lecture code on their own
machines. If you do not have access to a laptop on which you can install
\texttt{python3}, please contact the professor and/or TA for assistance.
Only \texttt{python3} will be used in this course.}

\hypertarget{course-requirements}{%
\section{Course Requirements}\label{course-requirements}}

\begin{longtable}[]{@{}lc@{}}
\toprule
\textbf{Assignment} & \textbf{Percentage of Grade}\tabularnewline
\midrule
\endhead
Participation & 5\%\tabularnewline
Coding Discussion & 15\%\tabularnewline
Problem sets & 40\%\tabularnewline
Final Project & 40\%\tabularnewline
\bottomrule
\end{longtable}

\textbf{Preparation and Participation} (5\%): It is imperative that
students keep up with the asynchronous materials and attend synchronous
lectures on time. See the ``Participation'' section in the Course
Policies section for more details.

\textbf{Coding Discussions} (15\%): Some weeks there will be a coding
problem/prompt/dataset to explore pushed to the class Github repository.
Students will be required to submit an original response to the prompt.
A point will be awarded for (1) submitting on time and (2) the quality
of the submission. ``Quality'' is defined as an \emph{original},
well-commented, and clear solution/analysis (i.e.~the solution follows
the guidelines required for assignments - see below).\footnote{Students
  are encouraged to generate a response before looking at the responses
  of their peers. If a student appears to have copied an answer of
  another student, we'll examine the time stamp and only award a quality
  point to the first entry of the timeline of responses that appear
  duplicative.}

All submission must be posted Sunday 11:59 PM (EST). The schedule for
the coding assignments are listed below.

\begin{longtable}[]{@{}ccc@{}}
\toprule
No. & Coding Discussion Week & Date Assigned\tabularnewline
\midrule
\endhead
1 & Week 2 & September 2\tabularnewline
2 & Week 3 & September 9\tabularnewline
3 & Week 6 & September 30\tabularnewline
4 & Week 10 & October 28\tabularnewline
5 & Week 11 & November 4\tabularnewline
6 & Week 12 & November 11\tabularnewline
\bottomrule
\end{longtable}

The goal of the coding discussions is to apply a concept learned during
the week in a way that helps build a greater level of programming
fluency. Programming skills are honed through active usage and
repetition. Learning to read other people's code and detecting issues is
vital to successful collaborations in applied work. The point is not to
be ``right'' necessarily but rather to try, learn, and collaborate.

\textbf{Problem Sets} (40\%): Students will be assigned four problem
sets over the course of the semesters. While you are encouraged to
discuss the problem sets with your peers and/or consult online
resources, \textbf{the finished product must be your own work}. The goal
of the assignment is to reinforce the student's comprehension of the
materials covered in each section. All assignments will be posted
Wednesday afternoon by 6:00PM (i.e., by the scheduled end of class) on
the class Canvas site for the weeks marked on the syllabus. Problem sets
are due on the date and time posted on Canvas and must be submitted on
Canvas. Generally, a week will be allotted to complete each assignment.
\textbf{\emph{Late assignments will be penalized a letter grade for
every day they are overdue}}.

The assignments can be in the form of a Jupyter Notebook
(\texttt{.ipynb}) or R Markdown (\texttt{.Rmd}). Student's must submit
completed assignments as a rendered \texttt{.html} file to Canvas on the
assigned due date. All assignment submissions must adhere to the
following guidelines:

\begin{itemize}
\item
  \begin{enumerate}
  \def\labelenumi{(\roman{enumi})}
  \tightlist
  \item
    all code must run;
  \end{enumerate}
\item
  \begin{enumerate}
  \def\labelenumi{(\roman{enumi})}
  \setcounter{enumi}{1}
  \tightlist
  \item
    solutions should be readable
  \end{enumerate}

  \begin{itemize}
  \tightlist
  \item
    Code should be thoroughly commented (the Professor/TA should be able
    to understand the codes purpose by reading the comment),
  \item
    Coding solutions should be broken up into individual code chunks in
    Jupyter/R Markdown notebooks, not clumped together into one large
    code chunk (See examples in class or reach out to the TA/Professor
    if this is unclear),
  \item
    Each student defined function must contain a doc string explaining
    what the function does, each input argument, and what the function
    returns;
  \end{itemize}
\item
  \begin{enumerate}
  \def\labelenumi{(\roman{enumi})}
  \setcounter{enumi}{2}
  \tightlist
  \item
    Commentary, responses, and/or solutions should all be written in
    Markdown and should contain no grammatical or spelling errors;
  \end{enumerate}
\item
  \begin{enumerate}
  \def\labelenumi{(\roman{enumi})}
  \setcounter{enumi}{3}
  \tightlist
  \item
    All mathematical formulas should be written in LaTex;
  \end{enumerate}
\item
  \begin{enumerate}
  \def\labelenumi{(\alph{enumi})}
  \setcounter{enumi}{21}
  \tightlist
  \item
    All solutions must be completed in Python 3.
  \end{enumerate}
\end{itemize}

The follow schedule lays out when each assignment will be assigned.

\begin{longtable}[]{@{}ccc@{}}
\toprule
Assignment & Date Assigned & Date Due\tabularnewline
\midrule
\endhead
No.~1 & September 16 & September 23\tabularnewline
No.~2 & October 7 & October 14\tabularnewline
No.~3 & October 14 & October 21\tabularnewline
No.~4 & November 18 & November 25\tabularnewline
\bottomrule
\end{longtable}

\textbf{Final Project} (40\%): Data science is an applied field and
therefore, it is important that you understand how to conduct a complete
analysis from collecting data, to cleaning and analyzing it, to
presenting your findings. Toward the end of the semester, you will
complete an independent data science project, \emph{applying concepts
learned throughout the course}. The project is composed of three parts:
a 2 page project proposal, an in-class presentation, and a 12-page
project report. Due dates and breakdowns for the project are as follows:

\begin{longtable}[]{@{}llll@{}}
\toprule
\textbf{Requirement} & \textbf{Due} & \textbf{Length} &
\textbf{Percentage}\tabularnewline
\midrule
\endhead
Project Proposal & October 28 & 2 pages & 5\%\tabularnewline
Presentation & December 2 & 7 minutes & 10\%\tabularnewline
Project Report & December 19 & 12 pages & 25\%\tabularnewline
\bottomrule
\end{longtable}

\textbf{\emph{Students will use Git/Github for version control to track
progress made on their analysis.}} Each student will be required to
create a public Github repository and use it to track progress made on
the project. Failure to version control one's work on the project could
result in a deduction in points on all components of the project.

Details regarding each aspect of the project will be posted on the
course website leading up to the first due date (i.e.~the Project
Proposal). Until then, we will not discuss the project in class. The
reason for this is that students need to reach a basic level of data
competency before thinking through a project idea. Thus, discussion of
the final project and the development of a project proposal will align
with the final portion of the class.

\hypertarget{grading}{%
\section{Grading}\label{grading}}

Course grades will be determined according to the following scale:

\begin{longtable}[]{@{}ll@{}}
\toprule
Letter & Range\tabularnewline
\midrule
\endhead
A & 95\% -- 100\%\tabularnewline
A- & 91\% -- 94\%\tabularnewline
B+ & 87\% -- 90\%\tabularnewline
B & 84\% -- 86\%\tabularnewline
B- & 80\% -- 83\%\tabularnewline
C & 70\% -- 79\%\tabularnewline
F & \textless{} 70\%\tabularnewline
\bottomrule
\end{longtable}

\hypertarget{how-to-succeed}{%
\section{How to Succeed}\label{how-to-succeed}}

\begin{itemize}
\item
  \textbf{Come Prepared.}

  \begin{itemize}
  \item
    Do the readings. Think about the readings on their own terms, but
    also in terms of how the concepts apply to things you're interested
    in.
  \item
    It is expected that students bring their computers to class to
    partake in computational activities or play with coding being
    discussed in class. Moreover, students should have all relevant
    software up and running on their machines.
  \end{itemize}
\item
  \textbf{Ask Questions.}

  \begin{itemize}
  \tightlist
  \item
    Formulating a question helps you engage with the material much more
    deeply. If you have a question, it's almost certain that others do
    too; asking a question will not only help yourself, but you will
    help others. Most importantly, asking questions helps keep the class
    on track. If there are lots of questions, we'll slow down and get
    things figured out. If there are few questions, we'll charge ahead.
  \end{itemize}
\item
  \textbf{Collaborate.}

  \begin{itemize}
  \tightlist
  \item
    Utilize \textbf{the class
    \href{ppol670introt-5qu2508.slack.com}{slack channel}} to pose any
    questions, insights, coding problems and concerns. The channel will
    offer an open forum to communicate, collaborate, and collectively
    problem solve.
  \end{itemize}
\item
  \textbf{Start homework early.}

  \begin{itemize}
  \tightlist
  \item
    Sometimes the data doesn't cooperate, or there is an error in your
    code that will take you awhile to figure out and debug. You don't
    want to find this out at 11pm the night the homework is due. Also,
    the more you are doing homeworks on time, the more you will be able
    to follow the lectures.
  \end{itemize}
\item
  \textbf{Try doing it the hard way.}

  \begin{itemize}
  \tightlist
  \item
    A core factor in the success of a data scientist is being able to
    explain how an algorithm or analysis was constructed, not just use
    software. In this class, where possible, build from scratch rather
    than an overly convenient library. This will allow you to become
    more creative down the line.
  \end{itemize}
\end{itemize}

\hypertarget{course-policies}{%
\section{Course Policies}\label{course-policies}}

\hypertarget{participation}{%
\subsubsection{Participation}\label{participation}}

Participation is required in this course. I define participation as:

\begin{itemize}
\tightlist
\item
  Attending synchronous lecture components over Zoom.
\item
  Completing the readings and asynchronous materials prior to the
  synchronous lecture.
\item
  Asking questions and participating in class.
\item
  During synchronous lectures, cameras are active at all times.
\item
  Paying attention to the professor during lecture
\item
  Engage in break-out group discussions when assigned.
\item
  Responding to questions asked during synchronous sessions. 
\end{itemize}

I reserve the right to deduct participation points from students who are
not participating as expected.

\hypertarget{communication}{%
\subsubsection{Communication}\label{communication}}

\begin{itemize}
\item
  Class-relevant and/or coding-related questions, \texttt{Slack} is the
  preferred method of communication. Please use the general or the
  relevant channel for these questions.
\item
  For private questions concerning the class, email is the preferred
  method of communication. All email messages must originate from your
  Georgetown University email account(s). Please use a professional
  salutation, proper spelling and grammar, and patience in waiting for a
  response. The professor reserves the right to not respond to emails
  that are drafted inappropriately. \textbf{\emph{Please email the
  professor and the TA directly rather than through the Canvas messaging
  system.}} Emails sent through CANVAS will be ignored.
\item
  I will respond to all emails/slack questions \textbf{\emph{within 24
  hours}} of being sent during a weekday. \textbf{\emph{I will not
  respond to emails/slack sent late Friday (after 5:00 pm) or during the
  weekend until Monday (9:00 am)}}. Please plan accordingly if you have
  questions regarding current or upcoming assignments. Please address
  the professor and TA by their last name unless stated otherwise.
\end{itemize}

\hypertarget{electronic-devices}{%
\subsubsection{Electronic Devices}\label{electronic-devices}}

When meeting in-person: the use of laptops, tablets, or other mobile
devices is permitted \emph{only for class-related work}. Audio and video
recording is not allowed unless prior approval is given by the
professor. Please mute all electronic devices during class.

\hypertarget{assignments-and-late-work}{%
\subsubsection{Assignments and Late
Work}\label{assignments-and-late-work}}

Assignments should be clear, legible, and submitted in the required
format. Writing assignments will be graded on the basis of content,
logic, analysis, mechanics, organization, and research. Due dates for
all assignments will be posted on Canvas and are non-negotiable.
Exceptions to this policy will be made only under extremely unusual
circumstances and will require valid documentation from the student.
\textbf{\emph{Late problem sets will be penalized a letter grade per
day.}}

\hypertarget{proof-of-diligent-debugging}{%
\subsubsection{Proof of Diligent
Debugging}\label{proof-of-diligent-debugging}}

When reaching out to the professor or teaching assistant regarding a
technical question, error, or issue you \textbf{\emph{must}} demonstrate
that you made a good faith effort to debugging/isolate your problem
prior to reaching out. In as concise a way as possible, send a record of
what you tried to do. \textbf{\emph{The professor/TA is a resource of
last resort}}. As software is continually being refined in data science
and new approaches continually emerge and changing, learning how to
frame your question and find a similar solution online is a key tool for
success in this domain. If you make a diligent effort beforehand to
solve your problem, we will do the same in trying to help you figure out
a solution.

\hypertarget{use-of-class-materials}{%
\subsubsection{Use of Class Materials}\label{use-of-class-materials}}

Increasingly, with the proliferation of certain websites, questions
about the ownership of course materials have arisen (and Georgetown is
actively working on policies to address these concerns). I consider my
syllabus, lectures, videos, handouts, problem sets, and problem set
answers to be my intellectual property. I respectfully request that you
refrain from sharing my materials in any electronic (or paper) format.
You are welcome to save my lectures for your own use, but they should
not be posted anywhere. Sharing notes, on an occasional basis, with
others in the class is fine as long as they are not posted. Students
found in breach of this policy will fail the course.

\hypertarget{academic-resource-centerdisability-support}{%
\subsubsection{Academic Resource Center/Disability
Support}\label{academic-resource-centerdisability-support}}

If you believe you have a disability, then you should contact the
Academic Resource Center (\url{arc@georgetown.edu}) for further
information. The Center is located in the Leavey Center, Suite 335
(202-687-8354). The Academic Resource Center is the campus office
responsible for reviewing documentation provided by students with
disabilities and for determining reasonable accommodations in accordance
with the Americans with Disabilities Act (ASA) and University policies.
For more information, go to
\url{http://academicsupport.georgetown.edu/disability/}.

\hypertarget{important-academic-policies-and-academic-integrity}{%
\subsubsection{Important Academic Policies and Academic
Integrity}\label{important-academic-policies-and-academic-integrity}}

McCourt School students are expected to uphold the academic policies set
forth by Georgetown University and the Graduate School of Arts and
Sciences. Students should therefore familiarize themselves with all the
rules, regulations, and procedures relevant to their pursuit of a
Graduate School degree. The policies are located at:
\href{The\%20follow\%20schedule\%20lays\%20out\%20when\%20each\%20assignment\%20will\%20be\%20assigned}{http://grad.georgetown.edu/academics/policies/}.

\hypertarget{plagiarism}{%
\subsubsection{Plagiarism}\label{plagiarism}}

Plagiarism is the intentional or unintentional presentation of another
person's idea or product as one's own. Plagiarism includes, but is not
limited to the following: copying verbatim all or part of someone else's
written work; using phrases, charts, figures, illustrations, code, or
mathematical / scientific solutions without citing the source;
paraphrasing ideas, conclusions, or research without citing the source;
and using all or part of a literary plot, poem, film, musical score, or
other artistic product without attributing the work to its creator. In
technology, plagiarism is the verbatim use of a code chunk from a peer
or third party website to complete an assignment questions. Students can
avoid unintentional plagiarism by following carefully accepted scholarly
practices. Students who plagiarize will receive a 0 on the plagiarized
assignment and may fail the course, if deemed necessary.

\hypertarget{provosts-policy-accommodating-students-religious-observances}{%
\subsubsection{Provosts Policy Accommodating Students Religious
Observances}\label{provosts-policy-accommodating-students-religious-observances}}

Georgetown University promotes respect for all religions. Any student
who is unable to attend classes or to participate in any examination,
presentation, or assignment on a given day because of the observance of
a major religious holiday (see below) or related travel shall be excused
and provided with the opportunity to make up, without unreasonable
burden, any work that has been missed for this reason and shall not in
any other way be penalized for the absence or rescheduled work. Students
will remain responsible for all assigned work. Students should notify
professors in writing at the beginning of the semester of religious
observances that conflict with their classes. The Office of the Provost,
in consultation with Campus Ministry and the Registrar, will publish,
before classes begin for a given term, a list of major religious
holidays likely to affect Georgetown students. The Provost and the Main
Campus Executive Faculty encourage faculty to accommodate students whose
bona fide religious observances in other ways impede normal
participation in a course. Students who cannot be accommodated should
discuss the matter with an advising dean.

\hypertarget{statement-on-sexual-misconduct}{%
\subsubsection{Statement on Sexual
Misconduct}\label{statement-on-sexual-misconduct}}

Please know that as a faculty member I am committed to supporting
survivors of sexual misconduct, including relationship violence, sexual
harassment and sexual assault. However, university policy also requires
me to report any disclosures about sexual misconduct to the Title IX
Coordinator, whose role is to coordinate the University's response to
sexual misconduct.

Georgetown has a number of fully confidential professional resources who
can provide support and assistance to survivors of sexual assault and
other forms of sexual misconduct. These resources include:

\begin{verbatim}
Associate Director
Jen Schweer, MA, LPC
Health Education Services for Sexual Assault Response and Prevention 
(202) 687-0323
jls242@georgetown.edu
\end{verbatim}

\begin{verbatim}
Erica Shirley
Trauma Specialist
Counseling and Psychiatric Services (CAPS) 
(202) 687-6985
els54@georgetown.edu
\end{verbatim}

More information about campus resources and reporting sexual misconduct
can be found at \url{http://sexualassault.georgetown.edu}.

\hypertarget{course-calendar}{%
\section{Course Calendar}\label{course-calendar}}

\begingroup\fontsize{13}{15}\selectfont

\resizebox{\linewidth}{!}{
\begin{tabu} to \linewidth {>{\centering\arraybackslash}p{.5in}>{\raggedright\arraybackslash}p{1in}>{\raggedright\arraybackslash}p{3in}>{\raggedright\arraybackslash}p{2in}>{\centering\arraybackslash}p{1in}}
\toprule
Week & Date & Topic & Assignment & Coding Discussion\\
\midrule
\rowcolor{gray!6}  1 & 26-Aug & Introductions, Installations, and IDEs &  & \\
2 & 2-Sep & Version Control, Workflow, and Reproducibility &  & X\\
\rowcolor{gray!6}  3 & 9-Sep & Object-Oriented Programming in Python &  & X\\
4 & 16-Sep & Introduction to Algorithms and Data Structures & Assignment 1 Assigned & \\
\rowcolor{gray!6}  5 & 23-Sep & From Nested Lists to Data Frames & Assignment 1 Due & \\
\addlinespace
6 & 30-Sep & Approaches to Data Manipulation in Python &  & X\\
\rowcolor{gray!6}  7 & 7-Oct & Data Visualization and Exploration & Assignment 2 Assigned & \\
8 & 14-Oct & Drawing from (Un-)Structured Data Sources & Assignment 2 Due; Assignment 3 Assigned & \\
\rowcolor{gray!6}  9 & 21-Oct & Introduction to Statistical Learning & Assignment 3 Due & \\
10 & 28-Oct & Continuous Outcomes and Linear Regression & Project Proposals Due & X\\
\addlinespace
\rowcolor{gray!6}  11 & 4-Nov & Probability, Bayes Theorem, and Classification &  & X\\
12 & 11-Nov & Non-parametric Approaches to Supervised Learning &  & X\\
\rowcolor{gray!6}  13 & 18-Nov & Interpretable Machine Learning & Assignment 4 Assigned & \\
- & 25-Nov & Thanksgiving Break & Assignment 4 Due & \\
\rowcolor{gray!6}  14 & 2-Dec & Project Presentations &  & \\
\addlinespace
Final & 19-Dec & Final Project Due (5:00 PM EST) &  & \\
\bottomrule
\end{tabu}}
\endgroup{}

\textbf{IMPORTANT: This syllabus is subject to change and may be amended
throughout the course to reflect any changes deemed necessary by the
professor. Any changes will be announced in class or over Slack.}

\end{document}
